\documentclass[12pt]{article}
\usepackage[margin=0.75in]{geometry}
\usepackage{graphicx}
\usepackage{multicol}
\usepackage{float}
\usepackage{upgreek}
\usepackage{amsmath}

% for dots in TOC:
\usepackage{tocloft}
\renewcommand{\cftsecleader}{\cftdotfill{\cftdotsep}}

\setlength{\parindent}{0mm}

\begin{document}

{\centering
\LARGE Physics I-II Lab Manual \par
}
\hfill \break

{\centering
\large Preface \par
}
\hfill \break \vspace{-4mm}

This is the lab manual for Physics 2211L, 2212L, 1111L, and 1112L taught by Dr. Nathan Harrison.
In some labs you will have to copy and paste small amounts of code;
it is recommended that you copy from the the original \LaTeX \ source, \textit{not} from the PDF file which often leads to formatting errors.
Another common mistake is to not unzip the Java-based simulations;
if the simulation opens but the screen is blank it is because you didn't unzip the file.
\vfill
\textcopyright \ Nathan Harrison 2018
\pagebreak \clearpage

\tableofcontents
\pagebreak \clearpage

\section{Surface area, volume, and density with uncertainty}

You will be studying the properties of 3 objects - a rectangular prism, a sphere, and a (partially) hollow cylinder.
Don’t forget to use the appropriate number of significant figures, units, and error bars.
When taking a measurement, use half of the smallest reading on the instrument as the error bar;
for example, for a ruler with mm markings, the error bar should be 0.5 mm.
Use the ``$\pm$'' symbol to report the error bar; for example, Width = $3.55 \pm 0.05$ cm.
\begin{enumerate}
\item Determine the mass of each object by using the electronic balance in the front of the room and record the results.
\item Determine the relevant dimensions of each object, measure them with the calipers, and record the results. 
\item Use your measurements to calculate the surface area, volume, and density of each object. Remember to propagate the error and report the error bar.
\item Use available resources to identify the material of the object. Site your source(s).
\end{enumerate}

\pagebreak \clearpage

\section{Constant acceleration 1-D motion and data fitting}

\underline{\textbf{Part 1}} \par
Suppose you throw a ball straight up with initial height 1.0 m and initial velocity 2.1 m/s.
Create a plot of the ball's height as a function of time (assume $t_{initial} = 0$).
\hfill \break
\hfill \break
Example of how to plot in SageMath:
\begin{verbatim}
t = var("t")
p1 = plot(sin(10*t) + sin(10.1*t), (t, 15, 40), color="red")
g = Graphics()
g += p1
g.show()
\end{verbatim}
\hfill \break \vspace{-4mm}

\underline{\textbf{Part 2}} \par
Sketch a plot of the ball's position on the $y_1$ axis as a function of time.
Repeat for the $y_2$ axis.
\begin{figure}[H]
\includegraphics[scale=0.4]{figures/fitting1Dmotion/twoAxisFreeFall.png}
\end{figure}

\underline{\textbf{Part 3}} \par
Match each set of parameters below to one of the colored curves.
Briefly justify your answers.
\begin{itemize}
  \item $x_i = 0 \ m$, $v_i = -5 \ m/s$, $a = 5 \ m/s^2$
  \item $x_i = 0 \ m$, $v_i = 2 \ m/s$, $a = -4 \ m/s^2$
  \item $x_i = 8 \ m$, $v_i = 4 \ m/s$, $a = 3 \ m/s^2$
\end{itemize}
\begin{figure}[H]
\includegraphics[scale=0.4]{figures/fitting1Dmotion/threeCurves.png}
\end{figure}

\underline{\textbf{Part 4}} \par
Now suppose you have some data collected from similar experiments on two other planets (different gravitational accelerations) (see Data section below).
Use this data to determine
(a) the gravitational acceleration of each planet,
(b) the orientation of the axis used in the experiment (up or down),
(c) the magnitude and direction (up or down) of the initial velocity, and
(d) the starting coordinate of the ball.
Use the following approach:
\begin{enumerate}
\item Download manfit-app.zip from https://github.com/naharrison/manual-fitter/releases, unzip it, and open the jar file.
\item From the unzipped folder, also open data/XYdata.txt and data/fitFunction.txt. Try to understand how the data in these files is being used by the application.
\item Copy the data below into XYdata.txt. Also modify fitFunction.txt to use a function that describes an object under constant acceleration: [0] + [1]*x + 0.5*[2]*x\string^2; the lines under the function are the minimum and maximum possible values for each parameter.
\item Restart the application and tune the parameters to determine the initial position, initial velocity, and acceleration.
\end{enumerate}

\pagebreak

\underline{\textbf{Data}} \par
\begin{multicols}{2}
\begin{verbatim}
Planet 1:
time (s)  y1 (m)
0.00      20.20
0.10      18.90
0.20      17.72
0.30      16.28
0.40      15.44
0.50      14.78
0.60      13.63
0.70      12.65
0.80      12.31
0.90      11.66
1.00      11.41
1.10      10.77
1.20      10.13
1.30      9.82
1.40      9.59
1.50      9.50
1.60      9.75
1.70      9.76
1.80      9.53
1.90      9.68
2.00      9.77
2.10      10.34
2.20      10.50
2.30      11.19
2.40      11.60
2.50      12.10
\end{verbatim}

\columnbreak

\begin{verbatim}
Planet 2:
time (s)  y2 (m)
0.00      -15.35
0.10      -14.17
0.20      -13.23
0.30      -12.16
0.40      -11.25
0.50      -10.41
0.60      -9.37
0.70      -7.82
0.80      -5.67
0.90      -3.65
1.00      -2.49
1.10      0.09
1.20      2.12
1.30      6.17
1.40      8.58
1.50      9.04
1.60      11.96
1.70      15.47
1.80      17.06
1.90      19.96
2.00      23.01
2.10      26.00
2.20      30.62
2.30      32.30
2.40      34.93
2.50      40.45
\end{verbatim}
\end{multicols}

\pagebreak \clearpage

\section{Projectile Motion}

In this lab you will be using the Interactive Physics (IP) software to simulate various kinds of two-dimensional motion and comparing the results to calculations.
Note that you can work from home through your web browser by going to my.ung.edu $\rightarrow$ Remote Access $\rightarrow$ Virtual Lab $\rightarrow$ Download Client $\rightarrow$ HTML 5 Browser $\rightarrow$ VMware Horizon HTML Access and logging in.
As always, do not copy exactly the examples given in these instructions.
Note that it may be possible to use the “guess and check” method in this lab to get approximately correct results; do not do this!
You must show all of your calculations.

Setup:
\begin{enumerate}
\item Open IP
\item Click View $\rightarrow$ Workspace and check ``Grid Lines'' and ``X,Y Axes''
\item Pan the screen so the origin is near the bottom left
\end{enumerate}

\underline{\textbf{Part 1}} \par
\begin{itemize}
\item Use the rectangle tool to create a floor and anchor it.
\item Use the circle tool to create a small circle, drag it to some height above the floor, and give it some velocity in the y-direction (right click the circle to adjust its values); keep $v_x = 0$ for now.
\item Create a second circle at a different height.
Calculate the necessary initial y-velocity so the two circles hit the floor at the same time.
Give the second circle this initial y-velocity.
\item Use the Windows ``Snipping Tool'' to take a screen shot of your initial conditions. (see fig.~\ref{fig:fig1})
\item Select both circles and click Windows $\rightarrow$ Appearance and check ``Track outline''.
Note: additional tracking settings are under World.
Run the simulation until both circles hit the floor.
Take another screenshot to show that your calculations were correct. (see fig.~\ref{fig:fig2})
\item Restart the simulation (clear the tracks by clicking World $\rightarrow$ Erase Track).
Give the circles some random (but reasonable) x-velocity to show they still hit the floor at the same time.
Run the simulation and take another screenshot. (see fig.~\ref{fig:fig3})
\end{itemize}
%
\begin{figure}[H]
\centering
\includegraphics[scale=0.35]{figures/projectileMotion/fig1.png}
\caption{Initial conditions.}
\label{fig:fig1}
\end{figure}
%
\begin{figure}[H]
\centering
\includegraphics[scale=0.35]{figures/projectileMotion/fig2.png}
\caption{1-D motion.}
\label{fig:fig2}
\end{figure}
%
\begin{figure}[H]
\centering
\includegraphics[scale=0.35]{figures/projectileMotion/fig3.png}
\caption{Projectile motion.}
\label{fig:fig3}
\end{figure}

\underline{\textbf{Part 2}} \par
\begin{itemize}
\item Using similar techniques as in Part 1, design an experiment where a projectile collides with a freely falling object.
Show relevant screen shots. (see figures~\ref{fig:fig2_1}~and~\ref{fig:fig2_2})
\end{itemize}
%
\begin{figure}[H]
\centering
\includegraphics[scale=0.35]{figures/projectileMotion/fig2_1.png}
\caption{Part 2 initial conditions.}
\label{fig:fig2_1}
\end{figure}
%
\begin{figure}[H]
\centering
\includegraphics[scale=0.35]{figures/projectileMotion/fig2_2.png}
\caption{Part 2 result.}
\label{fig:fig2_2}
\end{figure}

\pagebreak \clearpage

\section{Forces}

In this lab you will investigate several applications of Newton's second law by calculating expected outcomes and comparing to (simulated) experiments.

Some comments:
\begin{enumerate}
\item When doing calculations, you treat objects like points, so in your simulations you should make your objects relatively small to achieve better results.
\item Each object in I.P. has coefficients of friction under its properties. When two objects are in contact, I.P. calculates the friction based on the smaller of the two values.
\item A stopwatch feature is available in I.P. under Measure $\rightarrow$ Time.
\end{enumerate}

\underline{\textbf{Part 1}} \par
For each diagram below, use a ruler to accurately draw the x and y components of the given vector.
Also use a ruler and protractor to measure the magnitudes $v$, $v_x$, and $v_y$, and at least one relevant angle.
Also write down the sign (positive or negative) of each component.
Confirm your measurements using the Pythagorean Theorem and trigonometry functions.
%
\begin{figure}[H]
\includegraphics[scale=0.70]{figures/forces/fig1.png}
\end{figure}

\underline{\textbf{Part 2}} \par
Create an inclined ramp experiment.
Choose reasonable values for the angle of incline and the coefficients of friction (should be non-zero but small enough that the block slides). 
%
\begin{figure}[H]
\includegraphics[scale=0.70]{figures/forces/fig2a.png}
\end{figure}
%
Calculate how long the block should take to travel some distance along the ramp.
Run your experiment and compare results.
Make sure to turn tracking on and to record relevant information and screenshots.
%
\begin{figure}[H]
\includegraphics[scale=0.70]{figures/forces/fig2b.png}
\end{figure}

\underline{\textbf{Part 3}} \par
Create the experiment shown below.
Let the bottom block be $m_1$ and the top block be $m_2$.
The floor is a frictionless surface, however, some friction exists between $m_1$ and $m_2$.
A constant horizontal force, $F$, is applied to $m_1$.
If the force is small, then the friction between the two blocks keeps them together; but if the force is large enough, then block 1 slides out from underneath block 2.
Play around with some settings to get a feel for this.
Next, choose some reasonable values for $m_1$, $m_2$, and $F$.
Calculate the coefficient of static friction such that the blocks are right on the brink of slipping, and set this value in the experiment.
Run the experiment several times, changing slightly a few values to confirm that you found the correct coefficient of friction.

Note: here we are not very interested in the coefficient of kinetic friction, but to avoid bugs in IP make sure it is equal to the coefficient of static friction.
%
\begin{figure}[H]
\includegraphics[scale=0.70]{figures/forces/fig3.png}
\end{figure}

\pagebreak \clearpage

\section{Energy}

In this lab you will explore how different forms of energy govern the dynamics of a given system.

Use Interactive Physics to set up the following experiment:
%
\begin{figure}[H]
\includegraphics[scale=0.70]{figures/energy/fig1.png}
\end{figure}
%
Note that the diagram shows a suggested coordinate system choice, but you can define your own if you prefer.
Choose some random but reasonable values for k, m1, and m2 and run the simulation and observe the results.
Before proceeding, discuss with your lab partners why you think the systems behaves the way it does (you don't need to be too quantitative yet).

\bigskip

\underline{\textbf{Part 1}} \par
Determine the theoretical maximum displacement of m1 with respect to its initial position.
Use the simulation to confirm your result.

\bigskip

\underline{\textbf{Part 2}} \par
In terms of k, m1, m2, x, v, and g, use energy conservation to write down a general equation relating these quantities at any arbitrary time t.
Choose a random value of t and use IP to find the relevant quantities at this time.
Plug these values into your equation.
Is the equation (approximately) satisfied?

\pagebreak \clearpage

\section{Work and Power}

\underline{\textbf{Part 1}} \par
Use a constant horizontal force to push a block ($v_i = 0$) up a ramp that has friction.
Choose and record all relevant values.
Identify all forces present, label each one as conservative or non-conservative, and calculate the amount of work each one does on the block.
Use these work values to calculate the expected final velocity after some time interval.
Compare this calculation to the final velocity measured by IP.

\begin{figure}[H]
\includegraphics[scale=0.50]{figures/workPower/fig1.png}
\end{figure}


\underline{\textbf{Part 2}} \par
In this lab you will conduct a real life experiment to calculate your (or a lab mate's) maximum power output, with an uncertainty.

\begin{enumerate}
\item Write down your weight (with uncertainty) and convert to mass (in kg).
\item Gather relevant supplies (a ruler, stopwatch/smartphone, paper, pencil, and calculator) and go as a class to the staircase next to the Nesbitt building.
\item Use your ruler to measure the height of one stair. Also count the number of stairs to get the total elevation change from the bottom to the top. Make sure to propagate the error.
\item Have one person control the stop watch while another person runs up the hill next to the staircase as fast as possible. Record the time taken, with uncertainty. The runner and timer should coordinate a strategy so that the runner's velocity at the bottom is approximately equal to the velocity at the top; that way the change in energy is only due to the change in height. Include your strategy in your lab report. Although your stopwatch probably has very good precision, you may want to assign a larger time uncertainty due to the human error of starting and stopping at the right time.
\item Write down the relevant equations from class and calculate your power with uncertainty. Give your answer both in W and hp.
\end{enumerate}

In the event of bad weather, you may do the following experiment instead.
Here you will measure your maximum power by jumping as high as you can.
In this part you will measure your maximum power by jumping as high as you can.
\begin{enumerate}
\item Use a meter stick to measure the three relevant heights shown in the figure (with uncertainty).
\item Also write down your weight (converted to kg, with uncertainty).
\item Assume constant acceleration and force, and use kinematics to calculate relevant velocities and time intervals. Also calulate the uncertainty in these quantities using error propagation.
\item Finally, calculate your power during the jump with uncertainty. Record your result in Watts and horsepower.
\end{enumerate}

\begin{figure}[H]
\includegraphics[scale=0.50]{figures/workPower/fig2.png}
\end{figure}

\pagebreak \clearpage

\section{Momentum}

Suppose you are standing on a frictionless sled ($M_{you+sled} = 100$ kg) moving at 0.5 m/s.
You also have 22 blocks of mass 12 kg (i.e. $M_{total} = M_{you+sled}$ + 22(12 kg)).
You can boost your velocity by throwing the blocks horizontally at velocity 11 m/s in the opposite direction of your motion.
Create the following two plots: velocity vs the number of blocks thrown, and momentum vs the number of blocks thrown.
\hfill \break

Run the following simulation to test your results:
\begin{itemize}
\item Make sure you have Java JDK 1.8 installed
\item Download rocket-app.zip from https://github.com/naharrison/discrete-rocket/releases
\item Unzip the file and open rocket.jar
\item Click the green button to throw a block
\end{itemize}

\pagebreak \clearpage

\section{Torque}

\begin{verbatim}
https://phet.colorado.edu/sims/html/balancing-act/latest/balancing-act_en.html
Click "Game"
Complete all 4 levels - take screenshots of your results
\end{verbatim}

\pagebreak \clearpage

\section{Angular Kinematics}

In this lab you will explore the relationships between mass distributions, torques, angular momentum, and angular kinematics.
\hfill \break

Open Interactive Physics, turn gravity off, and create a T-shaped rotational object as shown.
Keep the thickness reasonably small.
You will have to attach the object to an anchor using a ``pin joint'' and attach the two parts of the T together with ``rigid joints.''
%
\begin{figure}[H]
\includegraphics[scale=0.70]{figures/angularKinematics/figure1.png}
\end{figure}
%

\underline{\textbf{Part 1}} \par
Calculate the moment of inertia of the object about the base of the T.
Use the torque tool to apply a torque to the base of the T.
Measure the angular acceleration of the object for at least 4 values of torque.
Create a plot of $\alpha$ vs $\uptau$ and use this data to extract the object's moment of inertia.
Compare the result to your calculation.
\hfill \break

\underline{\textbf{Part 2}} \par
For a fixed value of torque, measure $\omega_i$ and $\omega_f$ over some time interval $\Delta t$.
Use this data to show that $\uptau = \Delta L / \Delta t$.
Repeat for a different value of torque.

\pagebreak \clearpage

\section{Orbits}

In this lab you will use SageMath to numerically calculate the paths for several different kinds of orbital motion.
\hfill \break

For simplicity, assume that for the given universe, planet, and unit system the product $GM_{planet} = 1$.
Also assume that the given satellite has a mass of 1.
Under these conditions the gravitational force is $F_g = 1/r^2$ and the requirement for circular motion is $F_c = v^2/r$.
(Do not make these assumptions outside of this lab.)
\hfill \break

\underline{\textbf{Part 0}} \par
Copy the orbit\_plot.sage script (below) and run the following example:
\begin{verbatim}
radius = var("radius")

blue_circle = circle((0, 0), 0.1, color="blue", fill=True, zorder=100, figsize=[5.5, 5.5])
planet_label = text("Planet", (0, 0), color="black", fontsize="large", zorder=101)

plot1 = orbit_plot(0.5, 0.0, 1.0, 1.0/(radius*radius), 0.01, 1.0, "green")
plot2 = orbit_plot(0.6, -1.0, 0.9, 1.0/(radius*radius), 0.01, 1.0, "red")

g = Graphics()
g += blue_circle
g += planet_label
g += plot1
g += plot2
g.set_axes_range(-0.8, 0.8, -0.8, 0.8)
g.show()
\end{verbatim}
%
\begin{figure}[H]
\includegraphics[scale=0.60]{figures/orbits/part0.png}
\end{figure}
%
Make sure you understand what each argument of orbit\_plot() means;
Refer to the table below:
\begin{enumerate}
  \item $x_i$ of the satellite ($y_i = 0$ by default)
  \item $v_{xi}$ of the satellite
  \item $v_{yi}$ of the satellite
  \item the gravitational force law, i.e. $F_g = 1/r^2$ (see above)
  \item the time interval between data points, make this smaller for more detail
  \item the path length at which the calculations stops, make this larger to see more of the path
  \item the color of the data points
\end{enumerate}
Adjust a few settings to see how they affect your resulting plot.
\hfill \break

\underline{\textbf{Part 1}} \par
Create two circular orbits of different radii.
Record your initial values and a screenshot of your plot.
%
\begin{figure}[H]
\includegraphics[scale=0.60]{figures/orbits/part1.png}
\end{figure}

\underline{\textbf{Part 2}} \par
a) Create an elliptical orbit by either
\begin{itemize}
\item making the velocity too large
\item making the velocity too small
\item making the velocity non-tangent to a circular path.
\end{itemize}
b) Create a case where the velocity is larger than the escape velocity.

Record your conditions and a screenshot of your plot.
%
\begin{figure}[H]
\includegraphics[scale=0.60]{figures/orbits/part2.png}
\end{figure}

\underline{\textbf{Part 3}} \par
Design your own universe with a different gravitational force law.
Experiment until you find an interesting result.
Record your conditions and a screenshot of your plot.
%
\begin{figure}[H]
\includegraphics[scale=0.60]{figures/orbits/part3.png}
\end{figure}

\underline{orbit\_plot.sage} \par
\begin{verbatim}
__author__  = "Nathan Andrew Harrison"
__license__ = "GPL-3.0"

# see Goldstein & Poole 3.7 and 3.11
#                       ^^^     ^^^^
#                       2 second order ODEs converted to 4 1st order ODEs


def orbit_plot(radius_i, vx_i, vy_i, force, stepsize, endpt, plotcolor):

  # variables
  theta, thetaPrime, radius, radiusPrime, time = var("theta, thetaPrime, radius, radiusPrime, time")
  theta_i = 0.0
  
  # calculations
  x_i = radius_i*cos(theta_i)
  y_i = radius_i*sin(theta_i)
  radiusPrime_i = 2.0*x_i*vx_i + 2.0*y_i*vy_i
  thetaPrime_i = ((vy_i*x_i - y_i*vx_i)/(x_i*x_i))/(pow(y_i/x_i, 2.0) + 1.0)
  
  sol = desolve_system_rk4([thetaPrime, -2.0*radiusPrime*thetaPrime/radius, radiusPrime, -force + radius*thetaPrime*thetaPrime], [theta, thetaPrime, radius, radiusPrime], ics=[0, theta_i, thetaPrime_i, radius_i, radiusPrime_i], ivar=time, end_points=[0, endpt], step=stepsize)
  
  
  pts = [[m*cos(k), m*sin(k)] for j, k, l, m, n in sol]
  
  return list_plot(pts, color=plotcolor)
\end{verbatim}

\pagebreak \clearpage

\section{Waves}

In this lab you will use software to model wave motion.
\hfill \break

\underline{\textbf{Part 1}} \par
At $t = 0$ a wave is given by
\begin{equation}
y(x) = 1.6e^{-0.75x^2} + e^{-0.5(x-3)^2}
\end{equation}
and moves to the right with speed $v$.
Choose a value for $v$ (anything besides 5 m/s since that is used in the example figure below) and create a plot showing the wave at $t =$ 0, 1, and 2 s.
\begin{figure}[H]
\includegraphics[scale=0.50]{figures/waves/fig1.png}
\end{figure}
Do this by modifying the SageMath starter code below.
\begin{verbatim}
y1, y2, y3, x = var("y1, y2, y3, x")

y1 = x + 1
y2 = x^2
y3 = x^3 - 1

p1 = plot(y1, (x, -2, 2), color="black")
p2 = plot(y2, (x, -2, 2), color="blue")
p3 = plot(y3, (x, -2, 2), color="red")

g = Graphics()
g += p1
g += p2
g += p3
g.show()
\end{verbatim}

\underline{\textbf{Part 2}} \par
Now create an animated version of the plot from part 1.
Save the animated .gif file and upload it to GitHub when you submit your lab.
Use the code below as a starting point.
\begin{verbatim}
traveling_wave = [plot(exp(-(x - t)^2), (-1, 4), ymin=0, ymax=2) for t in sxrange(0, 5, 0.1)]
animation = animate(traveling_wave)
animation.show(delay=10)
\end{verbatim}

\underline{\textbf{Part 3}} \par
Create an animation of a right-moving wave interfering with a left-moving wave.

\pagebreak \clearpage

\section{More Waves}

\underline{\textbf{Part 1}} \par

Choose reasonable values for velocity $v$, amplitude $A$, and wavelength $\lambda$ and reproduce the transverse traveling wave shown in lecture by using springs and masses in Interactive Physics.
Include several snapshots at different points in time.

\vspace{\baselineskip}

Tips: turn gravity off (optional); don't confuse the wave number $k$ with the spring constant $k_s$; consider the initial positions and velocities of each block.

\begin{figure}[H]
\includegraphics[scale=0.80]{figures/more-waves/fig1.png}
\end{figure}

Write down the function $f(x, t)$ that describes this wave and plot it in SageMath (see previous waves lab for examples).
Save the animated gif as part of your lab report.

\vspace{\baselineskip}

\underline{\textbf{Part 2}} \par

Create a standing wave in SageMath by summing together two traveling waves.
Save the animated gif as part of your lab report.


\pagebreak \clearpage

\section{Doppler Effect}

Supplies:
\begin{itemize}
\item wheel thingy
\item a few meter sticks
\item stop watch with lap function
\item pencil/paper
\end{itemize}

Instructions:
\begin{itemize}
\item Form into two group of 8-10 students.
\item Choose reasonable values for:
\begin{itemize}
\item wave propagation velocity, $v$
\item wavelength, $\lambda$
\item velocity of the observer, $v_s$
\end{itemize}
\item From these values calculate the expected unshifted period, $T$, and the observed (shifted) period $T^\prime$.
\item Have 5-6 students walk in a single file line with velocity $v$, separated by a distance $\lambda$, holding out their hands for a high-five.
\item Choose an observer from your group. The line of 5-6 students should then high-five the observer as they walk by. Do this twice - once with a stationary observer and again with the observer moving towards the line with velocity $v_s$.
\item While the experiment is running, have another student use a stopwatch to record the time interval between high-fives. Record several time intervals for each experiment and average them.
\item Compare your calculated $T$ and $T^\prime$ with your measured $T$ and $T^\prime$.
\end{itemize}

\pagebreak \clearpage

\section{Electric Potential and Fields}

In this lab you will study the electric fields and potentials produced by different charge distributions and how charges respond to the presence of these fields.

\vspace{\baselineskip}

\underline{\textbf{Part 1}} \par
1. Consider placing two point charges on an x-y plane, the first charge, $q_1$, at $(x_1, y_1)$, and the second charge , $q_2$, at $(x_2, y_2)$.

\vspace{\baselineskip}

2. Derive an expression for the electric potential and field at any arbitrary point $(x, y)$ in terms of $q_1, q_2, x_1, x_2, y_1$ and $y_2$.

\vspace{\baselineskip}

3. Choose some reasonable values for $q_1, q_2, x_1, x_2, y_1$ and $y_2$ and make a rough sketch of what you expect the electric potential/field to look like.
Your picture doesn't necessarily have to be correct, but you should include some justification for why your sketch looks the way it does.

\vspace{\baselineskip}

4. Use SageMath to draw the exact electric potential and field using the expression derived above.
Below is an example.

\begin{verbatim}
x, y = var("x y")
g = Graphics()
g += contour_plot(1.5 + 0.2*x*y, (x, -4, 4), (y, -4, 4), fill=False, cmap="jet", labels=True, contours=[0, 1, 2, 3, 4], label_fontsize=14)
g += plot_vector_field((y/2, -x/2) , (x, -4, 4), (y, -4, 4)) 
g.show()
\end{verbatim}

Your final result should look something like the figure below.
Note how you can clearly see the two point charges at $(1, 2)$ and $(2, 1.5)$.

\begin{figure}[H]
\includegraphics[scale=0.50]{figures/electric-potential-fields/fig1.png}
\end{figure}

\vspace{\baselineskip}

\underline{\textbf{Part 2}} \par

Go to http://www.physicsclassroom.com/PhysicsClassroom/media/interactive/Electric\%20Field\%20Hockey/index.html and complete levels 1, 2, 3, and 5.
Include screenshots in your lab report.

\pagebreak \clearpage

\input{electric-flux-and-path-integral.tex}
\section{Electric Flux and Path Summations}

\underline{\textbf{Part 1}} \par
Recall from class that when a point charge is located at the corner of a cube the electric flux through one of the opposite sides is $q / 24 \epsilon_0$.
Confirm this result (approximately) with the following procedure:

\begin{enumerate}
\item Divide the side of the cube into a 3x3 grid (as shown)
\item Calculate the electric field at the center of each grid square
\item Take the dot product of each of the 9 calculated electric fields with that grid square's area vector
\item Sum the 9 results above
\item Repeat with a 4x4 grid to show that more grid squares gives a better approximation to the exact result
\end{enumerate}

\begin{figure}[H]
\includegraphics[scale=1.0]{figures/electric-flux-and-path-integral/fig1.png}
\end{figure}

\underline{\textbf{Part 2}} \par

See the figure below.
Calculate the change in electric potential between points A $(0, 1)$ and B $(2, 1)$ along the three paths shown.
Show all work!
Note that the two parts of path (1) are parallel and perpendicular to the electric field.
Extra credit (5\%): repeat for the path defined by the parabola $y = 2x^2 - 4x + 1$.

\begin{figure}[H]
\includegraphics[scale=1.0]{figures/electric-flux-and-path-integral/fig2.png}
\end{figure}

\pagebreak \clearpage

\section{Resistor Circuits}

In this lab you will study the relationships between voltage, current, and resistance in simple DC circuits.

\vspace{\baselineskip}

Safety: electricity does have the ability to cause harm, but this lab should be very safe as long as you follow the instructions and don’t do anything stupid (e.g. take apart a power supply). 

\vspace{\baselineskip}

\underline{\textbf{Part 1}} \par
See https://phet.colorado.edu/sims/html/circuit-construction-kit-dc/latest/circuit-construction-kit-dc\_en.html for practice

\vspace{\baselineskip}

The DC power supplies at your lab stations are basically batteries with knobs so you can change the voltage.
Make sure your power supply is initially turned off (the switch may be in the back) and the knob is turned all the way down.
If it's not plugged in, plug it in and then turn it on and adjust the knob so the voltage is about 5V. 

\vspace{\baselineskip}

Take out your multi-meter and set it to voltage mode and measure the voltage of the power supply. It should read ~5V.

\vspace{\baselineskip}

The diagram below shows a standard breadboard layout:

\begin{figure}[H]
\includegraphics[scale=0.80]{figures/resistor-circuits/breadboard.png}
\end{figure}

The top 2 and bottom 2 rows (red (+) and blue (-)) are typically used for connecting to your power supply while the middle (black) is for plugging in circuit components.
Attempt to construct the following circuit:

\begin{figure}[H]
\includegraphics[scale=0.80]{figures/resistor-circuits/one-resistor.png}
\end{figure}

Have your instructor check your circuit before proceeding.
Next measure the voltage across the resistor and the current through the resistor.
Make sure your multi-meter is in the correct mode and that you connect it correctly.
Doing this incorrectly can blow a fuse in the multi-meter.
Note that measuring current requires you to break the circuit and put the meter in series with the resistor while measuring voltage requires you to put the meter in parallel with the resistor.
See below.

\begin{figure}[H]
\includegraphics[scale=0.55]{figures/resistor-circuits/V-I-measurement.png}
\end{figure}

\underline{\textbf{Part 2}} \par

With your 5V power supply, construct the circuit below.
You may use your multi-meter to measure the resistance of the resistors rather than rely on the colored bands.
Calculate and then measure the voltage across and current through each resistor.
Also calculate and measure the equivalent resistance of the 3 resistor system.
Remember to be careful not to blow a fuse in the multi-meter!

\begin{figure}[H]
\includegraphics[scale=0.90]{figures/resistor-circuits/mixed-circuit.png}
\end{figure}

\pagebreak \clearpage

\section{Logic Circuits}

\underline{\textbf{Part 1 A}} \par

Build circuits consisting of a power supply, resistors, switches, and LEDs that obey the following logic: NOT, AND, and OR.
Make sure the current through any LEDs remains between 10 and 20 mA.
Safe values for voltage and resistance are approximately 2 $<$ V $<$ 4 V (DC) and 200 $<$ R $<$ 500 $\Omega$.  

\vspace{\baselineskip}

\underline{\textbf{Part 1 B}} \par

Construct a truth table for the 2-input, 2-output circuit shown below.

\begin{figure}[H]
\includegraphics[scale=0.80]{figures/logic-circuits/fig1.png}
\end{figure}

\underline{\textbf{Part 2 A}} \par

Build NOT, AND, OR, and XOR circuits using a 4.5 V power supply, IC chips, resistors, and LEDs.
Note that the IC chips fit conveniently in the center of most breadboards as shown below and remember to use resistors in series with LEDs to keep the current between 10-20 mA.
Also note that a ``0'' input should be connected to ground (i.e. the negative side of your power supply), as opposed to being connected to nothing.
See the appendix for chip numbers.

\begin{figure}[H]
\includegraphics[scale=0.85]{figures/logic-circuits/fig2.png}
\end{figure}

Be very careful when removing the chips from the breadboards as the legs bend/break easily.
Avoid any unnecessary plugging and unplugging.
In order for the chips to function properly, pin 14 must be connected to +4.5 V and pin 7 must be connected to ground.
See below for pin numbers.

\vspace{\baselineskip}

The AND, OR, and XOR chips (below, left) each contain 4 gates while the NOT chips (below, right) contain 6 gates. 

\begin{figure}[H]
\includegraphics[scale=0.50]{figures/logic-circuits/fig3.png}
\end{figure}

In the above figures, A/B represent inputs and Y represents the output.
As an example, the figure below shows a more details picture of an OR chip.

\begin{figure}[H]
\includegraphics[scale=0.85]{figures/logic-circuits/fig4.png}
\end{figure}

\underline{\textbf{Part 2 B}} \par

Build the circuit below and confirm that it obeys XOR logic.

\begin{figure}[H]
\includegraphics[scale=0.85]{figures/logic-circuits/fig5.png}
\end{figure}

\underline{\textbf{Part 2 C}} \par

Fill out the truth table for the following 2-input, 2-output circuit.
In the last column, consider XY a 2-digit binary number and convert it to decimal (i.e. base 10).
Build and test the circuit.
What did you just create?

\begin{figure}[H]
\includegraphics[scale=0.80]{figures/logic-circuits/fig6.png}
\end{figure}

\begin{tabular}[H]{ | c | c | c | c | c | }
\hline
A & B & X & Y & $\text{XY}_{10}$ \\
\hline
0 & 0 & \  & \  & \  \\
\hline
0 & 1 & \  & \  & \  \\
\hline
1 & 0 & \  & \  & \  \\
\hline
1 & 1 & \  & \  & \  \\
\hline
\end{tabular}

\vspace{\baselineskip}

\underline{\textbf{Appendix}} \par

\vspace{\baselineskip}

OR: 296-1615-5-ND/SN74HCT32N \\
XOR: 296-4777-5-ND/SN74AHCT86N \\
AND: 296-1606-5-ND/SN74HCT08N \\
NOT: 296-1605-5-ND/SN74HCT04N

\pagebreak \clearpage

\section{Biot-Savart Law}

In this lab you will build a small model that represents the magnetic field produced by a cicular loop of current.

\vspace{\baselineskip}

Let the loop have radius $a$, the current have magnitude $I$, and consider the field at a distance $x$ away from the loop's center along its symmetry axis as shown.

\begin{figure}[H]
\includegraphics[scale=0.40]{figures/biot-savart/current-loop.png}
\end{figure}

If you consider the radius vectors connecting each small segment of the loop to the point of the field measurment, then a cone is formed.
Furthermore, according to the Biot-Savart Law, $d \vec{B} = \frac{\mu_0}{4 \pi} \frac{I d \vec{s} \times \hat{r}}{r^2}$, the fields produced by each small segment of the loop also form a second cone.

\begin{figure}[H]
\includegraphics[scale=0.40]{figures/biot-savart/cones.png}
\end{figure}

Construct the two cone system above out of paper using the following technique.
First note that if you cut a straight line from the base of a cone to its tip and unroll the cone, you get a pacman shape (this is very similar to unrolling a cylinder and getting a rectangle).

\begin{figure}[H]
\includegraphics[scale=0.40]{figures/biot-savart/pacman.png}
\end{figure}

Solve for $\theta_{pacman}^{(1)}$, $r_{pacman}^{(1)}$, $\theta_{pacman}^{(2)}$, and $r_{pacman}^{(2)}$ in terms of $a$, $x$, and $h$, where the superscripts refer to the first and second cones and $h$ is the height of the second cone (measured from the center of its base to its tip).
The value of $h$ is arbitrary, it depends on how long you want the field lines to be.
For this lab, let $a = 4.6 \ cm$, $x = 6.3 \ cm$, and $h = 4.8 \ cm$.

\vspace{\baselineskip}

Once you've solved for $\theta_{pacman}^{(1)}$, $r_{pacman}^{(1)}$, $\theta_{pacman}^{(2)}$, and $r_{pacman}^{(2)}$, use a protractor and compass to draw the pacmen on paper, and cut them out.
Tape the pieces together and draw a few relevant vectors on the cones.
If you did everything correctly, the two cones should form $90^\circ$ angles with each other.

\begin{figure}[H]
\includegraphics[scale=0.35]{figures/biot-savart/photo.png}
\end{figure}

\pagebreak \clearpage

\section{Stoke's Theorem and Divergence Theorem}

In this lab you will study the properties of the vector field
%
\begin{equation}
\label{eq:theVectorField}
\vec{v} = \frac{1}{\sqrt{0.2 + y^4 - 2xy^2 + x^2 + x^4 - 2x^2y + y^2}} \left[ \left(-y^2 + x \right) \hat{i} + \left( x^2 - y \right) \hat{j} + 0 \hat{k} \right]
\end{equation}
%
in the ranges $-2 < x < 2$, $-2 < y < 2$, $-2 < z < 2$ using Mathematica.
Begin by clearing any existing variables, defining the function, and defining variable ranges:
%
\begin{verbatim}
ClearAll["Global`*"];
denom[x_, y_] = Sqrt[0.2 + y^4 - 2*x*y^2 + x^2 + x^4 - 2*x^2*y + y^2];
vx[x_, y_] = (-y^2 + x)/denom[x, y];
vy[x_, y_] = (x^2 - y)/denom[x, y];
xmin = -2;
xmax = 2;
ymin = -2;
ymax = 2;
\end{verbatim}
%
Next, plot the vector field.
Since the field does not depend on $z$ and since the $z$-component is zero, a 2D plot will suffice.
Use VectorPlot and StreamPlot to get two different representations of the same function:
%
\begin{verbatim}
vp = VectorPlot[{vx[x, y], vy[x, y]}, {x, xmin, xmax}, {y, ymin, ymax}];
sp = StreamPlot[{vx[x, y], vy[x, y]}, {x, xmin, xmax}, {y, ymin, ymax}];
Grid[{{vp, sp}}]
\end{verbatim}
%
\begin{figure}[!h]
\centering
\includegraphics[scale=0.6]{figures/stokes-theorem/theVectorFieldPlots}
\caption{Plots of the vector field defined in equation~\ref{eq:theVectorField}.}
\label{fig:theVectorFieldPlots}
\end{figure}

\pagebreak

%%%%%%%%%%%%%%%%%%%%%%%%%%%%%%%%%%%%%%%%%%%
%%%%%%%%%%%%%%%%%%%%%%%%%%%%%%%%%%%%%%%%%%%

\underline{Part 1 - Stoke's Theorem}
\hfill \break

Stoke's Theorem states
%
\begin{equation}
\label{eq:stokesTheorem}
\oint_{Path} \vec{v} \cdot d\vec{s} = \int_{Surface} \left( \nabla \times \vec{v} \right) \cdot d\vec{A}.
\end{equation}
%
In this example, since the $z$-component of the field is zero, and since the field does not depend on z, Stoke's Theorem can be simplified to
%
\begin{equation}
\label{eq:stokesTheoremSimplified}
\oint_{Path} \vec{v} \cdot d\vec{s} = \int_{Surface} \left( \nabla \times \vec{v} \right)_z dA.
\end{equation}
%
Use Mathematica to calculate the curl of $\vec{v}$ (i.e. $\nabla \times \vec{v}$) and print out a few values at random points:
\begin{verbatim}
curlv[x_, y_, z_] = Curl[{vx[x, y], vy[x, y], 0}, {x, y, z}]
curlv[1, 1, 0]
curlv[1, 1, 0][[3]]
\end{verbatim}
notice that the $x$ and $y$ components of the curl of this vector field are zero.
Now make a plot of $\left( \nabla \times \vec{v} \right)_z$ as a function of $x$ and $y$ (remember, it doesn't depend on $z$):
\begin{verbatim}
dpz = DensityPlot[curlv[x, y, 0][[3]], {x, xmin, xmax}, {y, ymin, ymax},
	PlotLegends -> Automatic]
\end{verbatim}
%
\begin{figure}[!h]
\centering
\includegraphics[scale=0.7]{figures/stokes-theorem/curlzDensityPlot.png}
\caption{$\left( \nabla \times \vec{v} \right)_z$}
\label{fig:curlzDensityPlot}
\end{figure}
%
and compare figures~\ref{fig:theVectorFieldPlots} and \ref{fig:curlzDensityPlot}.
Notice that the curl is large in the region where the field is ``curly.''

Create a grid in the $x$-$y$ plane and use a loop to get the value of the curl at the grid points:
\begin{verbatim}
nxdiv = 10;
nydiv = 10;
For[ix = 0, ix < nxdiv, ix++;
 For[iy = 0, iy < nydiv, iy++;
  xval = xmin + (ix - 0.5)*(xmax - xmin)/nxdiv;
  yval = ymin + (iy - 0.5)*(ymax - ymin)/nydiv;
  curlval = curlv[xval, yval, 0][[3]];
  vals = {xval, yval, curlval};
  Print[vals];
  ]] 
\end{verbatim}
and use these results to approximate the right hand side of equation~\ref{eq:stokesTheoremSimplified}.

Finally, use a similar technique to approximate the left hand side of equation~\ref{eq:stokesTheoremSimplified}.
For example:
\begin{verbatim}
(* ccw loop, starting at top right *)
nStepsPerSide = 20;
(* top: *)
For[i = 0, i < nStepsPerSide, i++;
 xval = xmax - (i - 0.5)*(xmax - xmin)/nxdiv;
 yval = ymax;
 vDotDs = vx[xval, yval]*(-1)*((xmax - xmin)/nStepsPerSide);
 Print[vDotDs];
 ]
(* left side: *)
For[i = 0, i < nStepsPerSide, i++;
 xval = xmin;
 yval = ymax - (i - 0.5)*(ymax - ymin)/nydiv;
 vDotDs = vy[xval, yval]*(-1)*((ymax - ymin)/nStepsPerSide);
 Print[vDotDs];
 ]
etc...
\end{verbatim}
Once your program is fully written and debugged, increase $nxdiv$, $nydiv$, and $nStepsPerSide$ to $100$ to improve your accuracy.
Confirm Stoke's Theorem by comparing your two calculated numbers.

%%%%%%%%%%%%%%%%%%%%%%%%%%%%%%%%%%%%%%%%%%%
%%%%%%%%%%%%%%%%%%%%%%%%%%%%%%%%%%%%%%%%%%%

\hfill \break
\underline{Extra Credit - The Divergence Theorem}
\hfill \break

Use a similar approach to confirm the Divergence Theorem for this same vector field.

\pagebreak \clearpage

\section{Path of Least Time}

Consider a situation in which you have to rescue a drowning friend.
You want to get to him in the least amount of time, and since your velocity on land is larger than your velocity in water your path may not necessarily be a straight line. 
(This is an analogy for how light travels.)

\vspace{\baselineskip}

Part A - Derive a formula for finding the path of least time.

\vspace{\baselineskip}

Part B - Experimentally confirm your equation using the simulation found here: https://github.com/naharrison/path-of-least-time/releases

\vspace{\baselineskip}

Follow the instructions on that site to open the simulation – make sure to actually extract the files from the zip file.
Right-click to run/restart the simulation; left-click to move the 3 points of interest.
The relevant data will be printed to the screen in the following order:

\vspace{\baselineskip}

Starting point x

Starting point y

Transition point x

Transition point y

End point x

End point y

Elapsed time

\vspace{\baselineskip}

Note that the unit of distance is pixels and the unit of time is ms.
The velocity on land is 0.06 px/ms and the velocity in water is 0.015 px/ms. Also note that the origin is in the top left of the screen.


\pagebreak \clearpage

\section{Chaos}

In this lab you will study some interesting properties of chaotic systems.

\vspace{\baselineskip}

\underline{\textbf{Part 1}} \par
Draw 3 points on a piece of paper to form a triangle.
Label the points 0, 1, and 2.
Also randomly draw a 4th point - this will be the starting point of the experiment.
Next, randomly generate either a 0, 1, or 2 (e.g. by rolling a 3-sided die (or a more standard 6-sided die with a slightly more clever mapping algorithm)), the number generated will correspond to one of the corners of the triangle.
Make a new point halfway between the starting point and the corner of the triangle selected in the previous step; this point becomes the new starting point for the next iteration.
Repeat this process of making new points halfway between the last point and the corner of the triangle randomly selected many times.

\vspace{\baselineskip}

Do at least 10 points by hand until you get the idea.
Your final result should contain at least 2500 points.
If you are a student who insists that learning basic coding is not useful then you can do all 2500 points by hand.
Otherwise, try the following SageMath script (a few adjustments will be needed).

\begin{verbatim}
# Create a graphics object to display the points
g = Graphics()


# Add 3 points that make up a triangle
tri = [ (1, 1), (3, 1), (2, 3) ]
g += point(tri, color="red", size=35)


# Choose a random starting point
x = 2.6 
y = 1.3 
g += point((x, y), color="black", size=10)


# Generate random numbers (0, 1, or 2)
# corresponding to points on the triangle.
# For each iteration, move halfway from the 
# last point to the chosen triangle point.

nPts = 20
for i in range(0, nPts):
  rand = ZZ.random_element(0, 3)
  x = x + 0.25*(tri[rand][0] - x) # experiment with the fraction
  y = y + 0.25*(tri[rand][1] - y) # experiment with the fraction
  g += point((x, y), color="black", size=10)


g.show()
\end{verbatim}

% some interesting results:
% 0.25 total randomness
% 0.5 perfect fractal pattern
% 0.8 3 groups of 3 groups of 3 groups

\underline{\textbf{Part 2}} \par

This time choose (0, 0) as your starting point and let's call this point $(x_0, y_0)$.
The next point will be $(x_1, y_1)$, and then $(x_2, y_2)$, etc.
Just like before, each point will depend on the previous point and on some randomly generated number.
Imagine rolling a 100-sided die, with sides labeled 0-99.

\vspace{\baselineskip}

If you roll a 0, then

\begin{align}
x_{n+1} &= 0 \\
y_{n+1} &= 0.16 y_n
\end{align}

\vspace{\baselineskip}

If you roll 1-85, then

\begin{align}
x_{n+1} &= 0.85 x_n + 0.04 y_n \\
y_{n+1} &= -0.04 x_n + 0.85 y_n + 1.6
\end{align}

\vspace{\baselineskip}

If you roll 86-92, then

\begin{align}
x_{n+1} &= 0.2 x_n - 0.26 y_n \\
y_{n+1} &= 0.23 x_n + 0.22 y_n + 1.6
\end{align}

\vspace{\baselineskip}

If you roll 93-99, then

\begin{align}
x_{n+1} &= -0.15 x_n + 0.28 y_n \\
y_{n+1} &= 0.26 x_n + 0.24 y_n + 0.44
\end{align}

\vspace{\baselineskip}

Do at least 10 points by hand until you get the idea.
Use the SageMath starter code below to create an example with at least 20000 points (might take a few minutes to draw).

\begin{verbatim}
# Create a graphics object to display the points
g = Graphics()


# Set the starting point
x = 0.0 
y = 0.0 
g += point((x, y), color="black", size=1)


# Generate random numbers 0-99
# and follow the given algorithm

nPts = 20
for i in range(0, nPts):
  previousX = x 
  previousY = y 
  rand = ZZ.random_element(0, 100)

  if rand == 0:
    x = 0.0 
    y = 0.16*previousY

  if rand >= 1 and rand <= 85: 
    x =
    y =

  if rand >= 86 and rand <= 92: 
    x =
    y =

  if rand >= 93 and rand <= 99: 
    x =
    y =

  g += point((x, y), color="green", size=1)

g.show()
\end{verbatim}


\pagebreak \clearpage

%%  hyp_a = 0.8 # x-intercept of hyperbola
%%  hyp_c = 1.2 # x-value of hyperbola focal point
%%  
%%  ell_b = 1.5 # length of short radius for ellipse
%%  ell_c = 2.1 # distance between center and focal point for ellipse
%%  ell_th = 0.5 # theta rotation angle (radians) of ellipse (ccw)
%%  ell_a = sqrt(ell_b^2 + ell_c^2)
%%  
%%  # want the left focal point of the ellipse to match the left focal point of the hyperbola:
%%  ell_x = ell_c*cos(ell_th) - hyp_c # ellipse center x
%%  ell_y = ell_c*sin(ell_th) # ellipse center y
%%  
%%  ell_rfx = ell_x + ell_c*cos(ell_th) # ellipse right focal point x
%%  ell_rfy = ell_y + ell_c*sin(ell_th) # ellipse right focal point y
%%  ell_lfx = ell_x - ell_c*cos(ell_th) # ellipse left focal point x
%%  ell_lfy = ell_y - ell_c*sin(ell_th) # ellipse left focal point y
%%  
%%  var('x y')
%%  xmin = -2.5
%%  xmax = 3.5
%%  ymin = -1.5
%%  ymax = 3.5
%%  
%%  hyp = (x^2)/(hyp_a^2) - (y^2)/(hyp_c^2 - hyp_a^2) == 1
%%  hyp_plot = implicit_plot(hyp, (x, xmin, xmax), (y, ymin, ymax), color="red")
%%  
%%  ell = (((x - ell_x)*cos(ell_th) + (y - ell_y)*sin(ell_th))^2)/(ell_a^2) + (((x - ell_x)*sin(ell_th) - (y - ell_y)*cos(ell_th))^2)/(ell_b^2) == 1
%%  ell_plot = implicit_plot(ell, (x, xmin, xmax), (y, ymin, ymax))
%%  
%%  g = Graphics()
%%  g += hyp_plot
%%  g += ell_plot
%%  g += point((-hyp_c, 0), size=35, color="red", alpha=0.5)
%%  g += point((hyp_c, 0), size=35, color="red", alpha=0.5)
%%  g += point((ell_x, ell_y), size=35, alpha=0.5)
%%  g += point((ell_rfx, ell_rfy), size=35, alpha=0.5)
%%  g += point((ell_lfx, ell_lfy), size=35, alpha=0.5)
%%  g.show()

\section{Optics - Mirrors}

In this lab you will study the reflective properties of curved mirrors.

\hfill \break
\underline{Part 1 - Hyperbolic Mirrors}
\hfill \break

Consider the hyperbola
%
\begin{equation}
\label{eq:hyperbolaEq}
\frac{x^2}{0.8^2} - \frac{y^2}{1.2^2 - 0.8^2} = 1
\end{equation}
%
shown if figure~\ref{fig:hyperbolaPlot}.
Choose a light ray originating from the right-most region of the plot and heading towards the left focal point.
Show mathematically that when this ray is reflected by the hyperbolic mirror, it will pass through the second focal point.
Use a ruler to \textit{accurately} draw the path of this ray, stopping at the second focal point.
Draw the path for at least two more rays that are also initially headed towards the left focal point.
%
\begin{figure}[!h]
\centering
\includegraphics[scale=0.85]{figures/optics-mirrors/hyperbolaPlot.png}
\caption{The hyperbola defined in equation~\ref{eq:hyperbolaEq}.}
\label{fig:hyperbolaPlot}
\end{figure}

%%%%%%%%%%%%%%%%%%%%%%%%%%%%%%%%%%%%
%%%%%%%%%%%%%%%%%%%%%%%%%%%%%%%%%%%%

\hfill \break
\underline{Part 2 - Elliptical Mirrors}
\hfill \break

Consider the ellipse
%
\begin{equation}
\label{eq:ellipseEq}
\frac{x^2}{1.5^2 + 2.1^2} + \frac{y^2}{1.5^2} = 1
\end{equation}
%
shown if figure~\ref{fig:ellipsePlot}.
Choose a light ray originating from the right-most focal point.
Show mathematically that when this ray is reflected by the elliptical mirror, it will pass through the second focal point.
Use a ruler to accurately draw the path of this ray, stopping at the second focal point.
Draw the path for at least two more rays that also originate from the right-most focal point.
%
\begin{figure}[!h]
\centering
\includegraphics[scale=0.85]{figures/optics-mirrors/ellipsePlot.png}
\caption{The ellipse defined in equation~\ref{eq:ellipseEq}.}
\label{fig:ellipsePlot}
\end{figure}

%%%%%%%%%%%%%%%%%%%%%%%%%%%%%%%%%%%%
%%%%%%%%%%%%%%%%%%%%%%%%%%%%%%%%%%%%

\hfill \break
\underline{Part 3 - Mirror Combinations}
\hfill \break

As shown in figure~\ref{fig:both}, an ellipse and hyperbola share a focal point (left-most point).
Use a ruler, plus the rules of reflection from parts 1 and 2, to accurately draw the location where all\footnote{consider only rays that hit the ellipse first} rays originating from the right-most point intersect.
%
\begin{figure}[!h]
\centering
\includegraphics[scale=0.85]{figures/optics-mirrors/both.png}
\caption{An ellipse and a hyperbola sharing a focal point.}
\label{fig:both}
\end{figure}

\pagebreak \clearpage


\end{document}
