\section{Chaos}

In this lab you will study some interesting properties of chaotic systems.

\vspace{\baselineskip}

\underline{\textbf{Part 1}} \par
Draw 3 points on a piece of paper to form a triangle.
Label the points 0, 1, and 2.
Also randomly draw a 4th point - this will be the starting point of the experiment.
Next, randomly generate either a 0, 1, or 2 (e.g. by rolling dice), the number generated will correspond to one of the corners of the triangle.
Make a new point halfway between the starting point and the corner of the triangle selected in the previous step; this point becomes the new starting point for the next iteration.
Repeat this process of making new points halfway between the last point and the corner of the triangle randomly selected many times.

\vspace{\baselineskip}

Do at least 10 points by hand until you get the idea.
Your final result should contain at least 2000 points.
If you are a student who insists that learning basic coding is not useful then you can do all 2000 points by hand.
Otherwise, try the following SageMath script.

\begin{verbatim}
# Create a graphics object to display the points
g = Graphics()


# Add 3 points that make up a triangle
tri = [ (1, 1), (3, 1), (2, 3) ]
g += point(tri, color="red", size=35)


# Choose a random starting point
x = 2.6 
y = 1.3 
g += point((x, y), color="black", size=10)


# Generate random numbers (0, 1, or 2)
# corresponding to points on the triangle.
# For each iteration, move halfway from the 
# last point to the chosen triangle point.

nPts = 2500
for i in range(0, nPts):
  rand = ZZ.random_element(0, 3)
  x = x + 0.50*(tri[rand][0] - x) # experiment with the fraction
  y = y + 0.50*(tri[rand][1] - y) # experiment with the fraction
  g += point((x, y), color="black", size=10)


g.show()
\end{verbatim}

% some interesting results:
% 0.25 total randomness
% 0.5 perfect fractal pattern
% 0.8 3 groups of 3 groups of 3 groups

% TODO Add barnsley fern example

\pagebreak \clearpage
