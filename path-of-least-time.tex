\section{Path of Least Time}

Consider a situation in which you have to rescue a drowning friend.
You want to get to him in the least amount of time, and since your velocity on land is larger than your velocity in water your path may not necessarily be a straight line. 
(This is an analogy for how light travels.)

\vspace{\baselineskip}

Part A - Derive a formula for finding the path of least time.

\vspace{\baselineskip}

Part B - Experimentally confirm your equation using the simulation found here: https://github.com/naharrison/path-of-least-time/releases

\vspace{\baselineskip}

Follow the instructions on that site to open the simulation – make sure to actually extract the files from the zip file.
Right-click to run/restart the simulation; left-click to move the 3 points of interest.
The relevant data will be printed to the screen in the following order:

\vspace{\baselineskip}

Starting point x

Starting point y

Transition point x

Transition point y

End point x

End point y

Elapsed time

\vspace{\baselineskip}

Note that the unit of distance is pixels and the unit of time is ms.
The velocity on land is 0.06 px/ms and the velocity in water is 0.015 px/ms. Also note that the origin is in the top left of the screen.


\pagebreak \clearpage
